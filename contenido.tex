\part*{Parte}

\section*{Como escribir en \LaTeX}

\subsection*{Escritura de texto}

\subsubsection*{Textos matemático}

\subsubsection*{Codificación de imagenes}
\noindent - - -.

\bigskip

\part{Parte}

\section{Como escribir en \LaTeX}

\bigskip
\begin{center}
    \begin{tikzpicture}
  \path[mindmap,concept color=black,text=white]
    node[concept] {Computer Science}
    [clockwise from=0]
    child[concept color=green!50!black] {
      node[concept] {practical}
      [clockwise from=90]
      child { node[concept] {algorithms} }
      child { node[concept] {data structures} }
      child { node[concept] {pro\-gramming languages} }
      child { node[concept] {software engineer\-ing} }
    }  
    child[concept color=blue] {
      node[concept] {applied}
      [clockwise from=-30]
      child { node[concept] {databases} }
      child { node[concept] {WWW} }
    }
    child[concept color=red] { node[concept] {technical} }
    child[concept color=orange] { node[concept] {theoretical} };
\end{tikzpicture}
\end{center}

\newpage

\begin{center}
    \begin{tikzpicture}
        \path[mindmap, concept color=gray!10]
        node [concept] {Inteligencia Artificial}
        [clockwise from=0]
        child[concept color=gray]{node[concept]{Machine Learning}}
    \end{tikzpicture}
\end{center}

Referencia a Wikipedia ~\cite{eswiki:166659213}.