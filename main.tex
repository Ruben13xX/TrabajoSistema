\documentclass{article}
\usepackage{graphicx} % Required for inserting images
\usepackage{amsmath,mathtools}
\usepackage{float}
\usepackage[none]{hyphenat}

\usepackage[spanish]{babel}
\usepackage[utf8]{inputenc}
\usepackage[backend=biber]{biblatex}
\bibliography{referencias}

\usepackage{tikz}
\usetikzlibrary{mindmap}

\usepackage{circuitikz}
\ctikzset{bipoles/thickness=1.2}
\usetikzlibrary{shapes,arrows}

\usepackage{pgfplots}
\pgfplotsset{width=10cm,compat=newest}

\usepackage[papersize={216mm,271mm},tmargin=20mm,bmargin=20mm,
lmargin=20mm,rmargin=20mm]{geometry}


\title{Notas de la materia Fundamentos de Inteligencia Artificial} 
\author{Monzon Garcia Jesus Ruben Grupo 3-3}
\date{ \today }



\begin{document}
\sloppy
\maketitle
\begin{abstract}
    \noindent \LaTeX  Una vez terminadas las notas se realizara el resumen.
\end{abstract}

\section*{Panorama General(pipeline de extremo a extremo)}

\begin{enumerate}
    \item El \textbf{usuario} escribe un prompt (por ejemplo ¿Que es la IA?) y se envia via cliente (Navegador/API)
    \item \textbf{Transporte:} el prompt viaja por HTTPS/TLS al servidor modelo (request JSON).
    \item \textbf{Preprocesado/Moderacion:} el servidor puede ejecutar filtros de seguridad/moderacion y separacion de roles (system/user/assistant).
    \item \textbf{Tokenizacion:} el texto se convierte en una secuencia de tokens (subpalabras/bytes) y luego en enteros (ids).
    \item \textbf{Inferencia del modelo:}
        \begin{itemize}
            \item Los ids entran en la red neuronal (embedding lookup = añadir codificacion posicional)
            \item Se ejecutan N bloques de transformer (multi-head self-attention + MLPS)
            \item La ultima capa produce logits sobre el vocabulario
        \end{itemize}
    \item 
        
\end{enumerate}

\part*{Parte}

\section*{Como escribir en \LaTeX}

\subsection*{Escritura de texto}

\subsubsection*{Textos matemático}

\subsubsection*{Codificación de imagenes}
\noindent - - -.

\bigskip

\part{Parte}

\section{Como escribir en \LaTeX}

\bigskip
\begin{center}
    \begin{tikzpicture}
  \path[mindmap,concept color=black,text=white]
    node[concept] {Computer Science}
    [clockwise from=0]
    child[concept color=green!50!black] {
      node[concept] {practical}
      [clockwise from=90]
      child { node[concept] {algorithms} }
      child { node[concept] {data structures} }
      child { node[concept] {pro\-gramming languages} }
      child { node[concept] {software engineer\-ing} }
    }  
    child[concept color=blue] {
      node[concept] {applied}
      [clockwise from=-30]
      child { node[concept] {databases} }
      child { node[concept] {WWW} }
    }
    child[concept color=red] { node[concept] {technical} }
    child[concept color=orange] { node[concept] {theoretical} };
\end{tikzpicture}
\end{center}

\newpage

\begin{center}
    \begin{tikzpicture}
        \path[mindmap, concept color=gray!10]
        node [concept] {Inteligencia Artificial}
        [clockwise from=0]
        child[concept color=gray]{node[concept]{Machine Learning}}
    \end{tikzpicture}
\end{center}

Referencia a Wikipedia ~\cite{eswiki:166659213}.

\printbibliography


\end{document}
